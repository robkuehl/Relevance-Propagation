\documentclass[t, compress]{beamer}
% Packages
\usepackage[ngerman]{babel}
\usepackage[utf8]{inputenc}
\usepackage[T1]{fontenc}
\usepackage{graphicx}
\usepackage{grffile}
\usepackage{amsmath}
\usepackage{amsfonts}
\usepackage{amssymb}
\usepackage{hyperref}
\usepackage{xcolor}
\usepackage{caption}
\usepackage{tcolorbox}
\usepackage{amsthm}
%\usepackage{natbib}
\usepackage{verbatim}
\usepackage{listings}
\usepackage{subfig}
%\usepackage{minipage}

\usepackage{bibentry}

%%%%%%%%%%%%%%%%%%%%%%%%%%%%%%%%%%%%%
%EIGENE INCLUDES FUER DEN 3. VORTRAG:
%%%%%%%%%%%%%%%%%%%%%%%%%%%%%%%%%%%%%










%%%%%%%%%%%%%%%%%%%%%%%%%%%%%%%%%%%%%

%Eigene Befehle:
\newcommand{\rn}{\mathbb{R}}
\newcommand{\bx}{\boldsymbol{x}}
\newcommand{\bv}{\boldsymbol{v}}
% Die Beamervorlage einbinden
% Farben fuer die Uni Koeln
\definecolor{unikoelndarkblue}{rgb}{0.25, 0.47, 0.61}
\definecolor{unikoelngray}{rgb}{0.89, 0.89, 0.89}
\definecolor{unikoelnlightblue}{rgb}{0.91, 0.93, 0.95}
\definecolor{unikoelndarkgray}{rgb}{0.41, 0.33, 0.33}
\definecolor{unikoelnred}{rgb}{0.85, 0.16, 0.11}
\definecolor{unikoelngreen}{rgb}{0.16, 0.85, 0.11}

% Aussehen der Folien
\mode<presentation>
{
  \defbeamertemplate*{footline}{authordateframes}
  {
   \leavevmode%
   \hbox{%
   \begin{beamercolorbox}[wd=.5\paperwidth,ht=2.25ex,dp=1ex,left]{author in head/foot}%
   \hspace*{1em}\usebeamerfont{author in head/foot}%\insertshortauthor~
   \end{beamercolorbox}%
   \begin{beamercolorbox}[wd=.5\paperwidth,ht=2.25ex,dp=1ex,right]{date in head/foot}%
     \usebeamerfont{date in head/foot}\insertshortdate{}\hspace*{2em}
     \insertframenumber{}\hspace*{2ex} 
    \end{beamercolorbox}}%
   \vskip0pt%
  }
								
  \defbeamertemplate*{headline}{miniframeslogo}
  {%
    \begin{beamercolorbox}[colsep=1.5pt]{upper separation line head}%
    \end{beamercolorbox}
    \begin{beamercolorbox}{section in head/foot}
    \vskip2pt\insertsectionnavigationhorizontal{\paperwidth}{}{\hskip0pt plus1filll}\vskip2pt
    \end{beamercolorbox}%
    \begin{beamercolorbox}[colsep=1.5pt]{lower separation line head}
    \end{beamercolorbox}
  }
			
  \useoutertheme[footline=authorinstitute]{miniframes}
  \setbeamertemplate{mini frames}[box]

  \useinnertheme{rectangles}

  \setbeamertemplate{footline}[authordateframes]
  \setbeamertemplate{headline}[miniframeslogo]
}
			
% Etwas mehr Platz schaffen			
\addtobeamertemplate{frametitle}{}{\vspace*{-0.5\baselineskip}}

% Farbschema anpassen
\setbeamertemplate{section in head/foot shaded}[default][40]
\setbeamercolor*{section in head/foot}{fg=unikoelndarkgray, bg=unikoelngray}
\setbeamercolor*{subsection in head/foot}{fg=unikoelndarkgray, bg=unikoelngray}
\setbeamercolor*{upper separation line head}{fg=white, bg=unikoelndarkblue}
\setbeamercolor*{lower separation line title}{fg=white, bg=unikoelndarkblue}
\setbeamercolor*{author in head/foot}{fg=unikoelndarkgray, bg=unikoelngray}
\setbeamercolor*{date in head/foot}{fg=unikoelndarkgray, bg=unikoelngray}
\setbeamercolor*{palette primary}{fg=unikoelndarkblue,bg=white}
\setbeamercolor*{palette secondary}{fg=white, bg=unikoelnlightblue}
\setbeamercolor*{palette tertiary}{bg=unikoelndarkblue, fg=unikoelnlightblue}
\setbeamercolor*{structure}{fg=unikoelndarkblue}
\setbeamercolor{titlelike}{parent=pallette primary, fg=unikoelndarkblue, bg=white}
\setbeamercolor{frametitle}{fg=unikoelndarkblue, bg=white}
\setbeamercolor{block title}{fg=unikoelndarkblue, bg=white}
\setbeamercolor{block body}{fg=unikoelndarkgray, bg=white}
\setbeamercolor{block body}{fg=unikoelndarkgray, bg=white}
\setbeamercolor{block title example}{fg=white, bg=unikoelndarkblue}
\setbeamercolor{block body example}{bg=unikoelngray}						
\setbeamercolor{block title alerted}{fg=white, bg=unikoelnred}
\setbeamercolor{block body alerted}{bg=unikoelngray}
\setbeamercolor{alerted text}{fg=unikoelnred}
\setbeamercolor{example text}{fg=unikoelndarkblue}
                
% Fix fuer komisch ausgerichtete Spalten in Frames mit top-align
\makeatletter      
 \define@key{beamerframe}{t}[true]{% top
   \beamer@frametopskip=.2cm plus .5\paperheight\relax%
   \beamer@framebottomskip=0pt plus 1fill\relax%
   \beamer@frametopskipautobreak=\beamer@frametopskip\relax%
   \beamer@framebottomskipautobreak=\beamer@framebottomskip\relax%
   \def\beamer@initfirstlineunskip{}%
 }
\makeatother  

% Ein paar nuetzliche Beamermakros

\newenvironment{itemblock}[1]{%
	\begin{block}{#1}
		\begin{itemize}
}{%
		\end{itemize}
	\end{block}
}

\newenvironment{alitemblock}[1]{%
	\begin{alertblock}{#1}
		\begin{itemize}
}{%
		\end{itemize}
	\end{alertblock}
}

\newenvironment{exitemblock}[1]{%
	\begin{exampleblock}{#1}
		\begin{itemize}
}{%
		\end{itemize}
	\end{exampleblock}
}

		\newenvironment{enumblock}[1]{%
	\begin{block}{#1}
		\begin{enumerate}
}{%
		\end{enumerate}
	\end{block}
}

\newenvironment{alenumblock}[1]{%
	\begin{alertblock}{#1}
		\begin{enumerate}
}{%
		\end{enumerate}
	\end{alertblock}
}

\newenvironment{exenumblock}[1]{%
	\begin{exampleblock}{#1}
		\begin{enumerate}
}{%
		\end{enumerate}
	\end{exampleblock}
}

\newenvironment{mycolumns}{%
	\vspace*{-0.5\baselineskip}
	\begin{columns}[T, onlytextwidth]
}{%
	\end{columns}
}

\newcommand{\str}{\structure}

% Der eigentliche Vortrag  
\begin{document}
\section{Einleitung}
\begin{frame}{Aufgaben}
\begin{large}
\begin{center}
\vspace*{1cm}
\begin{minipage}{0.9\textwidth}
\begin{enumerate}
\item \textbf{Arbeit an einem CNN für den Pascal VOC 2012 Datensatz fortsetzen}
\vspace*{10pt}
\item \textbf{Implementierung des Ansatz der Deep Taylor Decomposition für DNN (insbesondere CNN)}
\vspace*{0pt}
\item \textbf{Vergleich LRP $\leftrightarrow$ Deep Taylor Decomposition}
\end{enumerate}
\end{minipage}
\end{center}
\end{large}
\end{frame}
\section{Implementierung eines CNN für den Pascal VOC}
\begin{frame}{Implementierung eines CNN für den Pascal VOC}
\begin{enumerate}
\item Das VGG-Modell
\item Finetuning
\item Regularisierung
\item Ein eigener Model Checkpoint
\item Vergleich von Ergebnissen
\end{enumerate}
\end{frame}

\begin{frame}{Das VGG-Modell}
\includegraphics[scale=0.3]{./grafiken_robin/vgg16.png}
\end{frame}

\begin{frame}{Finetuning}
Wir entfernen die Dense Layer aus dem bereits trainierten VGG16 und trainieren diese neu\\
\vspace*{0.5cm}
\includegraphics[scale=0.4]{./grafiken_robin/finetuning.png}
\end{frame}

\begin{frame}{Regularisierung}
\begin{itemize}
\item Hier besoners wichtig weil: 
\begin{enumerate}
\item der Datensatz klein ist
\item die Klassen ungleich viele Bilder enthalten
\end{enumerate}
\item Methoden:
\begin{enumerate}
\item Dropout
\item BatchNormalization
\item Data Augmentation
\item Sample anderer Klassen
\end{enumerate}
\end{itemize}
\end{frame}

\begin{frame}{Regularisierung}
\begin{itemize}
\item Data Augmentation
\begin{itemize}
\item Idee: Erweitern des Datensatzes um Generalisierungseigenschatfen und Performance des NN zu verbessern
\item Umsetzung: Zufällige Modifzierzungen der Bilder im Datenstrom zur Laufzeit
\item Implmentierung: mittels der Klasse ImageDataGenerator von Keras
\end{itemize}
\end{itemize}
\end{frame}

\begin{frame}{Regularisierung}
\begin{itemize}
\item Data Augmentation
\begin{itemize}
\item Beispiele:
\end{itemize}
\end{itemize}
\begin{figure}
\stackunder[5pt]{\includegraphics[scale=0.3]{./grafiken_robin/horizontal_flip.png}}{Horizontal Flip}%
\hspace{1cm}%
\stackunder[5pt]{\includegraphics[scale=0.3]{./grafiken_robin/brightness.png}}{Brightness Range}
\caption{}
\end{figure}
\footnote{Quelle: https://machinelearningmastery.com/how-to-configure-image-data-augmentation-when-training-deep-learning-neural-networks/}
\end{frame}

\begin{frame}{Sample anderer Klasses}
\begin{itemize}
\item Ein NN muss lerenen was zu einer Klasse gehört und was nicht
\item Problem: Falsche Entscheidungen für eine Klasse anhand von Merkmalen die Häufig mit dieser Klasse auftreten
\item Idee: Hinzufügen von Bildern die keine der zu trainierenden Klassen enhalten
\end{itemize}
\end{frame}

\begin{frame}{Ein eigener Model Checkpoint}
\begin{itemize}
\item Idee: Speichere das Model nicht zum Zeitpunkt minimalen Fehlers sondern anhand spezieller Metriken
\begin{enumerate}
\item Precision: 
\begin{align*}
&\frac{true\_positives}{true\_positives + false\_positives}\\
&\text{Welcher Anteil positiv vorhergesagter Label war korrekt?}
\end{align*}
\item Recall:
\begin{align*}
&\frac{true\_positives}{true\_positives + false\_negatives}\\
&\text{Welcher Anteil positiver Label wurde korrekt vorhergesagt?}
\end{align*}
\end{enumerate}
\end{itemize}
\end{frame}

\begin{frame}{Ein eigener Model Checkpoint}
\begin{itemize}
\item Ziel: beide möglichst groß mit ähnlicher Größenordnung
\item Implementierung: Speichere das Modell falls
\begin{enumerate}
\item mind. eine Metrik sich verbessert hat wärend die andere nicht schlechter geworden ist
\item die Metriken im Mittel besser geworden sind und ihr Abstand sich verringert hat
\end{enumerate}
\end{itemize}
\end{frame}

\begin{frame}{Vergleich der Ergebnisse}
\begin{itemize}
\item Das Modell für 2 Klassen
\item Klassen = Person, Pferd
\includegraphics[scale=0.06]{./grafiken_robin/person_horse_plots.png}
\end{itemize}
\end{frame}

\begin{frame}{Vergleich der Ergebnisse}
\begin{itemize}
\item Das Model für 5 Klassen
\item Klassen = Katze, Esstisch, Person, Flugzeug, Flasche
\includegraphics[scale=0.06]{./grafiken_robin/several_classes_plots.png}
\end{itemize}
\end{frame}


\section{Taylor Decomposition}
\frame{\sectionpage}
\frame{\frametitle{Taylor Decomposition - Rückblick}
\begin{itemize}
\item Einfaches Netzwerk mit einem Hidden Layer, ReLU Aktivierung und Sum-Pooling als Output.
\item Zusätzliche Voraussetzung: $b_j \leq 0$. 
\begin{figure}[H]
\includegraphics[width = 0.7 \textwidth]{grafiken_marc/simple_net.png}
\end{figure}
\item Für das Outputneuron $x_k$ gilt: $x_k = max(0,\sum_{j} x_j)$
%\item Gesucht ist eine Vorschrift, nach der die Relevanz $R_k$ auf die Neuronen der vorherigen Schicht verteilt wird.
\end{itemize}
}

\frame{\frametitle{Taylor Decomposition - Rückblick}
\begin{itemize}
\item Suche eine Nullstelle f\"ur die Taylorentwicklung von $R_k(\boldsymbol{x}) = \sum_j x_j$.

\begin{figure}
\includegraphics[width = 0.6\textwidth]{grafiken_marc/rel_prop_1.png}
\end{figure}
\pause
\item Wg. ReLU Aktivierung im vorherigen Layer und $\sum_j x_j \overset{!}{=} 0$ ist $\tilde{\boldsymbol{x}}=$\textbf{0} die einzige Nullstelle von $R_k$.
\item Wegen $R_j=\frac{\partial R_{k}}{\partial x_{j}}(x_j - \tilde{x}_j)=1 \cdot (x_j - 0)$ gilt also
\item $R_{j}=x_{j}= \max \left(0, \sum_{i} x_{i} w_{i j}+b_{j}\right)$
\end{itemize}
}

\frame{\frametitle{Taylor Decomposition - Generische Regel}
\begin{itemize}
%\item Führe nun für jede der $R_j$ Taylorentwicklung mit einem eigenen Entwicklungspunkt $\hat{x}^{(j)}$ durch.
\item Es gilt $R_j = x_j = \max \left(0, \sum_{i} x_{i} w_{i j}+b_{j}\right)$

\begin{figure}
\includegraphics[width = 0.6\textwidth]{grafiken_marc/rel_prop_2.png}
\end{figure}
%\pause
\item Unterscheide nun 2 Fälle:
\begin{enumerate}
\item $R_j=0$: Nicht aktivierte Neuronen sollen keine Relevanz zurückverteilen. Insbesondere gilt hier $\tilde{\bx}= \bx $.
\item $R_j>0$: Hierfür wird ein Richtungsvektor $\bv^{(j)}$ definiert. $\tilde{\bx}$ soll von der Form $\bx	+ t \cdot \bv^{(j)}$, mit $t \in \rn$ sein.
\end{enumerate}
\end{itemize}
}


\begin{frame}{Taylor - Entwicklungspunkt}
\begin{itemize}
\item Allgemeine Vorgehensweise: 
\item Durch Einsetzen von $\tilde{\bx}=\bx	+ t \cdot \bv^{(j)}$ in die Ebenengleichung $\sum_{i} \tilde{x}_{i} w_{i j}+b_{j} = 0$ lässt sich eine allgemeine Formel für $t$ finden.
\item Somit gilt: 
\begin{align*}
0 &= \sum_{i} \left(x_{i} + t v_i^{(j)} \right) w_{i j}+b_{j} \\
\Leftrightarrow -t& =\frac{\sum_{i} x_{i} w_{i j}+b_{j}}{\sum_{i} v_{i}^{(j)} w_{i j}}\\
\ \\
\Rightarrow x_i - \tilde{x_i} &= -t v_i^{(j)} = \frac{\sum_{i} x_{i} w_{i j}+b_{j}}{\sum_{i} v_{i}^{(j)} w_{i j}} v_i^{(j)}
\end{align*}
\end{itemize}
\end{frame}

\begin{frame}{Taylor - Entwicklungspunkt}
\begin{itemize}
\item Für die Umverteilung von der $l+1$-ten Schicht in die $l$-te Schicht gilt
\begin{align*}
R_i^l &= \sum_{j} R_{i \leftarrow j}^l =  \sum_{j} \frac{\partial R_{j}^{l+1}}{\partial x_{i}^l}
(x_i - \tilde{x_i}) \\
&= \sum_{j : R_j = 0} \frac{\partial R_{j}^{l+1}}{\partial x_{i}^l} \cdot 0 + \sum_{j: R_j>0} w_{ij} \frac{\sum_{i} x_{i} w_{i j}+b_{j}}{\sum_{i} v_{i}^{(j)} w_{i j}} v_i^{(j)} \\
&= \sum_{j} \frac{v_i^{(j)}  w_{i j}}{\sum_{i} v_{i}^{(j)} w_{i j}} R_j^{l+1}
\end{align*}
\item $\Rightarrow$ Allgemeine Formel in Abhängigkeit von $v_i^{(j)}$
\end{itemize}
\end{frame}


%\section{Part 2}
\begin{frame}{Aufgabe}
\begin{itemize}
\item Herleitung $z^B$ Regel und Implementierung der einzelnen Regeln
\item Vergleich der Ergebnisse
\end{itemize}
\end{frame}
%\section{Part 2}
\begin{frame}{Erweiterung auf tiefe Netze}
\begin{itemize}
\item Bei tiefen Netzen, insbesondere Convolutional Layern ist die Relevanzfunktion nicht unbedingt explizit angegeben.
\end{itemize}
\vspace*{20pt}
\begin{minipage}{0.42\textwidth}
\includegraphics[width=\textwidth]{grafiken_marc/simple_net.png}
\end{minipage}
\begin{minipage}{0.52\textwidth}
\includegraphics[width=\textwidth]{grafiken_marc/DNN.png}
\end{minipage}
\vspace*{10pt}
\begin{itemize}
\item Ein Feature kann vorhanden sein, aber bei der Bilderkennung keine Rolle spielen
\end{itemize}
\footnote{Grafik entnommen aus \ref{itm:Mont17}}
\end{frame}

\begin{frame}{Erweiterung auf tiefe Netze}
\begin{itemize}
\item Gesucht ist eine Approximation der Relevanz Funktion, die leicht zu analysieren ist.
\item Führe das Konzept \textbf{Relevanz-Modell} ein, um bei tieferen Netzen die Relevanzfunktion $R_j^{l+1}(\bx^{l})$ zu approximieren.
\item Nehme an, die Relevanzfunktion $R_j$ lässt sich in der Form $R_j = x_j \cdot c_j$ schreiben, wobei $c_j$ konstant ist.
\item Im Paper als "{}Training Free"{} Ansatz vorgestellt
\end{itemize}
\end{frame}

\begin{frame}{Erweiterung auf tiefe Netze}
\begin{itemize}
\item Nehme an, die Relevanzfunktion $R_j$ lässt sich in der Form $R_j = x_j \cdot c_j$ schreiben, mit $c_j$ konstant.
\item Betrachte die generische Redistributionsregel
\begin{align*}
R_i &= \sum_{j} \frac{x_{i} \cdot \rho(w_{i j})}{\sum_{i} x_{i} \cdot \rho(w_{i j})} R_{j}\\
& = x_{i} \sum_{j}  \frac{ \rho(w_{i j})}{\sum_{i} x_{i} \cdot \rho(w_{i j})} x_j \cdot c_j \\
&= x_i \sum_{j}   \rho(w_{i j}) \frac{\max \left(0, \sum_{i} x_{i} w_{i j}\right)}{\sum_{i} x_{i} \cdot \rho(w_{i j})} c_j
\end{align*}
\end{itemize}
\end{frame}

\begin{frame}{Erweiterung auf tiefe Netze}
\begin{itemize}
\item Nehme an, die Relevanzfunktion $R_j$ lässt sich in der Form $R_j = x_j \cdot c_j$ schreiben, mit $c_j$ konstant.
\item Betrachte die generische Redistributionsregel
\begin{align*}
R_i &= \sum_{j} \frac{x_{i} \cdot \rho(w_{i j})}{\sum_{i} x_{i} \cdot \rho(w_{i j})} R_{j}\\
& = x_{i} \sum_{j}  \frac{ \rho(w_{i j})}{\sum_{i} x_{i} \cdot \rho(w_{i j})} x_j \cdot c_j \\
&= x_i \underbrace{\sum_{j}   \rho(w_{i j}) \frac{\max \left(0, \sum_{i} x_{i} w_{i j}\right)}{\sum_{i} x_{i} \cdot \rho(w_{i j})} c_j}_{\colonapprox c_i}
\end{align*}
\end{itemize}
\end{frame}

\begin{frame}{Erweiterung auf tiefe Netze}
\begin{itemize}
\item Es gilt also
\begin{align*}
R_i = x_i \underbrace{\sum_{j}   \rho(w_{i j}) \frac{\max \left(0, \sum_{i} x_{i} w_{i j}\right)}{\sum_{i} x_{i} \cdot \rho(w_{i j})} c_j}_{\colonapprox c_i} = \sum_{j}    \frac{\rho(w_{i j}) \cdot x_i}{\sum_{i} x_{i} \cdot \rho(w_{i j})} R_j
\end{align*}
\item D.h. $R_i^l$ lässt sich wieder schreiben als $x_i^l \cdot c_i^l$, mit $c_i^l$ annähernd konstant.
\item Ausgehend von der letzten Schicht kann die Relevanz somit auch gemäß der hergeleiteten Regeln zurück zum Input verteilt werden.
\item Der Parameter $c_i^l$ wird dabei durch die betrachtete Regel "{}induktiv"{} gebildet.
\end{itemize}
\end{frame}


\begin{frame}{Zusammenhang mit LRP}
\begin{itemize}
\item Die klassische $LRP-0$ Formel kann als Deep Taylor Entwicklung im Nullpunkt gesehen werden
\item Betrachte O.B.d.A. ein Neuron $x_j$ mit $R_j > 0$. 
\item Die Suchrichtung $\bv$ ist hierbei der Punkt $\bx$ selbst, und somit gilt:
\pause
\begin{align*}
R_{i \leftarrow j}^l &=\frac{\partial R_{j}^{l+1}}{\partial x_{i}^l}
(x_i - \tilde{x_i})  = w_{ij} \cdot c_j \cdot (x_i - \tilde{x_i}) \\
&= w_{ij} \cdot c_j \cdot \frac{\sum_{i} x_{i} w_{i j}+b_{j}}{\sum_{i} x_{i} w_{i j}} x_i\\
&= \frac{x_i \cdot w_{ij}}{\sum_{i} x_{i} w_{i j}} x_j \cdot c_j
\end{align*}
\end{itemize}
\end{frame}


\begin{frame}{Zusammenhang mit LRP}
\begin{itemize}
\item Die klassische $LRP-0$ Formel kann als Deep Taylor Entwicklung im Nullpunkt gesehen werden
\item Betrachte O.B.d.A. ein Neuron $x_j^{l+1}$ mit $R_j > 0$. 
\item Die Suchrichtung $\bv$ ist hierbei der Punkt $\bx$ selbst, und somit gilt:
\begin{align*}
R_{i \leftarrow j}^l &=\frac{\partial R_{j}^{l+1}}{\partial x_{i}^l}
(x_i - \tilde{x_i})  = w_{ij} \cdot c_j \cdot (x_i - \tilde{x_i}) \\
&= w_{ij} \cdot c_j \cdot \frac{\sum_{i} x_{i} w_{i j}+b_{j}}{\sum_{i} x_{i} w_{i j}} x_i\\
&= \frac{x_i \cdot w_{ij}}{\sum_{i} x_{i} w_{i j}} \underbrace{ x_j \cdot c_j}_{ R_j }
\end{align*}
\end{itemize}
\end{frame}

\begin{frame}{Zusammenhang mit LRP}
\begin{itemize}
\item Für die totale Relevanz von $x_i$ gilt:

\begin{align*}
R_i^l = \sum_{j} R_{i \leftarrow j}^l &= \sum_{j} \frac{x_i \cdot w_{ij}}{\sum_{i} x_{i} w_{i j}} R_j^{l+1} = \sum_{j} \frac{z_{ij}}{\sum_{i} z_{i j}} R_j^{l+1}
\end{align*}
\item Mit der gleichen Vorgehensweise können auch die $LRP-\varepsilon$ Regel sowie $LRP-\gamma$ hergeleitet werden.
\item Deep Taylor Decomposition als Basis für \textit{LRP}
\item Wesentlicher Unterschied: Geforderte Konsistenz von Montavon et al. im Vergleich zu den \textit{LRP} Regeln
\end{itemize}
\end{frame}


\end{document}