\section{Deep Taylor Decomposition} 
\frame{\frametitle{Taylor Decomposition f\"ur Neuronale Netze}
\begin{itemize}
\item Problem bei der LRP bisher: Die Formeln machen Sinn und funktionieren, aber die theoretische Fundierung fehlt.
\item Mit theoretischer Fundierung sind evlt. allgemeinere Aussagen möglich.
\item Betrachte hierzu ein NN als Funktion $f: \rn^{p} \rightarrow \rn$, wobei $p$ die Inputgr\"osse bezeichnet. 
\item Nehme an, dass $f(x)> 0$ Evidenz f\"ur das gesuchte Objekt bedeutet.
\item Intuition: Verschiebe den Input, sodass $f(x)=0$. Dimensionen, in die "{}weiter"{} verschoben wurde, haben offensichtlich mehr zur Klassifizierung beigetragen.
\end{itemize}
}
\frame{\frametitle{Taylor Decomposition}
\begin{itemize}
\item Ansatz: Betrachte Taylor-Entwicklung von $f$. 
\item W\"ahle als Entwicklungspunkt $\hat{x}$ eine Nullstelle von $f$. 
\item Taylorentwicklung ist gegeben durch
\begin{align*}
f(\boldsymbol{x}) &= f(\hat{\boldsymbol{x}})+\left(\left.\frac{\partial f}{\partial \boldsymbol{x}}\right|_{\boldsymbol{x}=\hat{\boldsymbol{x}}}\right)^{\top} \cdot(\boldsymbol{x}-\hat{\boldsymbol{x}})+\varepsilon \\ 
&= 0+\sum_{p} \underbrace{\left|\frac{\partial f}{\partial x_{p}}\right|_{x=\hat{x}} \cdot\left(x_{p}-\hat{x}_{p}\right)}_{R_{p}(x)}+\varepsilon
\end{align*}
\item $R_p (x)$ stellt somit die Relevanz der $p-$ten Komponente des Inputs dar.
\item Genauigkeit h\"angt davon ab, ob Terme h\"oherer Ordnung vernachl\"assigt werden k\"onnen.
\end{itemize}
}

\frame{\frametitle{Taylor Decomposition}
\begin{itemize}
\item Problem: Wahl eines geeigneten Punkts $\hat{x}$ für das ganze Netzwerk.
\item Oft zu weit entfernt und gibt somit nicht so viele Informationen zur Entscheidungsfindung des Netzwerks.
\item Gradient Shattering (-> Montavon Vortrag, nochmal genauer nachschauen)
\item Abhilfe: Führe die Taylor Decomposition an jedem einzelnen Neuron mit einem eigenen Punkt $\hat{x}^{(j)}$ durch. 
\item Konzept der \textbf{Deep Taylor Decomposition} 
\end{itemize}
}

\frame{\frametitle{Deep Taylor Decomposition}
\begin{itemize}
\item Deep Taylor erklären
\end{itemize}
}